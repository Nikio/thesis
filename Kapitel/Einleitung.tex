\chapter{Einleitung}
\label{cha:Einleitung}
%Kontext 
\section{Kontext}
 
Ein grundlegendes Konzept für die Sicherung eines Netzwerkes ist das Wissen um den Zustand des Netzwerkes und seiner Elemente. In modernen Unternehmensnetzwerken mit tausenden Elementen, Schnittstellen und Abhängigkeiten ist es schwierig, jederzeit den Überblick zu behalten. Aus diesem Grund nutzen Unternehmen \textit{Security Information and Event Management (SIEM)} Systeme, die für die Analyse protokollierter Ereignisse und der Erkennung von potentiellen Bedrohungsszenarien verwendet werden.  

Die Industrie 4.0 ist ein Konzept, mit deren Hilfe die nächste Stufe der industriellen Revolution erreicht werden soll. Um dieses Ziel zu erreichen, sollen \textit{Smart Factories} erstellt werden, in denen Anlagen miteinander sowie Produkte mit den entsprechenden Anlagen kommunizieren können. Für dieses Vorhaben besteht ein steigendes Interesse an vertikaler und horizontaler Integration, d.h. der stärkeren Vernetzung der einzelnen Komponenten / Anlagen miteinander sowie der Anbindung industrieller Produktionsnetzwerke an das WAN (Internet). 
Dabei ist jedoch zu berücksichtigen, dass industrielle Produktionskomponenten, die auch mit dem Begriff \glqq cyber-physisches System\grqq{} bezeichnet werden, nicht für die Anbindung an externe Netzwerke entworfen wurden, sondern mit den Zielen Safety (Sicherheit der Mitarbeiter), Verlässlichkeit und Effizienz. Aus diesem Grund sind weder Komponenten noch die verwendeten Kommunikationstechnologien und -protokolle ausreichend gegen Bedrohungen aus dem Internet abgesichert. Dabei ist der Schutz dieser Systeme sogar besonders kritisch zu bewerten, da erfolgreiche digitale Angriffe sich aus der digitalen auf die physische Welt übertragen und materielle und finanzielle Schäden sowie eine Gefahr für die Beschäftigten der Anlagen darstellen können. 
Allerdings ist der Schutz dieser Systeme auf Grund mehrerer Faktoren problematisch. Die Zertifizierungsnotwendigkeit für die sichere Ausführung und der teils sehr alten Technologien und -standards ist nicht ohne weiteres möglich, da neben den Anforderungen der IT-Sicherheit für kritische Systeme vor allem auch die Anforderungen der Anlagentechnik gewahrt bleiben müssen. Die verwendeten speicherprogrammierbaren Steuerungen (SPS) sind für die Kontrolle der einzelnen Fertigungsprozessschritte auf eine sichere Funktionalität, jedoch nicht auf Manipulationssicherheit durch externe Quellen ausgerichtet. Basierend auf den Anforderungen innerhalb der industriellen Fertigungsnetzwerke sind Logik und Technologie auf Verlässlichkeit (d.h. Determinismus / Fehlerfreiheit), Reaktionsschnelligkeit und Robustheit gegen äußere physikalische Einflüsse der Anlagenumgebung ausgelegt. Diese Anforderungen und die ursprüngliche Entwicklung der Automation von logischen Steckkombinationen hin zu programmierbaren Steuerungen ergeben knappe Ressourcen im Sinne der Kommunikationsbandbreite, Speicher und Rechenleistung. Programmerweiterungen können durch zusätzliche Programmlogik Fehler und unberechenbares Verhalten des physikalischen Equipments mit sich bringen. Fehlende Standards ergeben zusätzlich eine Netzwerklandschaft, in der viele verschiedene herstellerspezifische Kommunikationsprotokolle, Geräte, Gerätetypen und Kommunikationstechnologien vertreten sind. Zwar werden Versuche und Anstrengungen unternommen Standards zu schaffen, allerdings ist der aktuelle Stand noch nicht soweit.

Diese Faktoren erschweren die Anwendung von typischen Sicherheitselementen wie den erwähnten SIEM Systemen ungemein. Neben der Notwendigkeit der Anpassung an die Anforderungen der Fertigungssysteme behindern die knappen Ressourcen und Limitierungen der Erweiterbarkeit und Aktualisierung der Anlagen die Anwendung der etablierten Sicherheitskonzepte aus den Unternehmensnetzwerken.

\section{Zielsetzung und Zweck}
In dieser Ausarbeitung soll diese Problematik im spezifischen Fall der SIEM Systeme untersucht werden. SIEM Systeme benötigen Informationen der Netzwerkelemente (sowohl der Endgeräte als auch der Netzwerkgeräte und der Kommunikationsmechanismen- und Schnittstellen). Ist die Verfügbarkeit der Informationen limitiert, kann das SIEM System auch nur im Rahmen der verfügbaren Informationen ungewöhnliches Verhalten der Systeme und Bedrohungsszenarien erkennen und die Sicherheitsoperatoren bei ihrer Arbeit unterstützen. Für die Absicherung eines industriellen Netzwerkes im Kontext der Industrie 4.0 ist eine Analyse der verfügbaren Informationsquellen notwendig. Basierend auf der bereits erläuterten Problematik wird jedoch die Hypothese aufgestellt, dass die typischen Komponenten und Technologien eines industriellen Netzwerkes kritische Informationslücken aufweisen, welche die Überwachung und Analyse kritischer Aspekte verhindern. Dies ist die Kernthese dieser Arbeit, die es zu untersuchen gilt. Für die Beantwortung dieser These werden die folgenden Forschungsfragen untersucht:
\begin{itemize}
\item Welche Informationen lassen sich jeweils aus Unternehmens- und Industrienetzwerken gewinnen?
\item Welche Unterschiede bestehen bzgl. der verfügbaren Informationen im Vergleich zwischen den Unternehmenstypen?
\item Sind die gefundenen Unterschiede als kritisch für die Überwachung industrieller Netzwerkes einzustufen?
\end{itemize}

Der Fokus für die Untersuchung der Unterschiede zwischen den Netzwerken soll auf das Fehlen von Informationskategorien in industriellen Netzwerken gesetzt werden. Die Elemente dieser konkreten potentiellen Untermenge der Unterschiede werden im Rahmen dieser Ausarbeitung als \textit{Informationslücken} bezeichnet.

\section{Vorgehen}
Als Startpunkt für die Suche nach den potentiellen Informationslücken in industriellen Netzwerken wird eine Referenzmenge benötigt. Diese Referenzmenge ergibt sich für SIEM Systeme aus bereits etablierten Implementierungen in Unternehmensnetzwerken. Aus diesem Grund ist der erste Schritt die Analyse der Informationsmenge eines Unternehmensnetzwerkes. Zu diesem Zweck werden zunächst Grundkenntnisse über das SIEM Konzept und dessen Anwendung sowie über Elemente und Technologien in Unternehmensnetzwerken und industriellen Netzwerken zur Verfügung gestellt. 

Die erste Herausforderung bei der folgenden Analyse ist die Eingrenzung der Informationsmenge. Durch eine sehr große Anzahl an unterschiedlichen Systemen, Protokollen, Schnittstellen und herstellerspezifischen Zusatzeigenschaften muss die Analyse auf ein handhabbares Modell beschränkt werden. Für die Analyse werden ein repräsentatives Modell definiert und die einzelnen Elemente genauer betrachtet. Ziel der Analyse ist es zum einen, die Informationsquellen, etwa in Form von Protokolldateien, zu identifizieren und diese in Kategorien auf Basis ihres Inhaltes einzuordnen. Dies ermöglicht die Eingrenzung und Fassung der Informationsmenge. 

Der gleiche Ansatz wird für die Analyse des industriellen Netzwerkes gewählt. Im Unterschied zu der vorherigen Analyse wird jedoch der Versuch unternommen, die Informationen den zuvor definierten Kategorien des Unternehmensnetzwerkes zuzuordnen. Basierend auf dieser Zuordnung entsteht eine notwendige Vergleichsbasis, auf der die beiden Netzwerkarten miteinander verglichen werden können.

Die Ergebnisse des Vergleiches müssen jedoch nicht zwangsläufig auf eine kritische Informationslücke schließen lassen. Durch die unterschiedlichen Zwecke und Fähigkeiten der Netzwerkelemente und Kommunikationstechnologien können Unterschiede in der Informationsmenge auch bspw. auf nicht unterstützte Funktionalitäten oder Dienste hinweisen. Um die Bedeutung von Unterschieden bzgl. der Notwendigkeit für die Überwachung zu ermitteln, müssen diese Unterschiede diesbzgl. bewertet werden. Die Bewertung wird anhand des populären Beispiels \textit{Stuxnet} für einen Angriff auf eine industrielle Anlage durchgeführt. Die Analyse der Aktionen des Stuxnet Angriffes erlaubt Rückschlüsse darauf, welche Informationen eines industriellen Netzwerkes notwendig sind, um Angriffe wie diesen zu erkennen. 
Ein Abgleich der gefunden Unterschiede mit den Analysepunkten des Stuxnetangriffes ergibt die Bedeutung der gefundenen Unterschiede.

Schlussendlich wird auf Basis dieser Erkenntnisse versucht, einen Lösungsansatz für eine konkrete Problemstelle unter bestimmten Annahmen zu formulieren.


\section{Struktur}
Die Struktur der Arbeit orientiert sich direkt an der beschriebenen Vorgehensweise. Kapitel 2 liefert die notwendigen Grundkenntnisse, die für das Verständnis der Untersuchung als notwendig erachtet werden. In den darauf folgenden Kapiteln werden die Analyse der Modelle des Unternehmensnetzwerkes (Kapitel 3) und des industriellen Netzwerkes (Kapitel 4) beschrieben. Die Vergleichsmethodik und die Ergebnisse des Vergleichs finden sich in Kapitel 5. Kapitel 6 beschreibt die Bewertung der Vergleichsergebnisse. Kapitel 7 beschreibt einen Lösungsansatz für eine ausgewählte Informationslücke.