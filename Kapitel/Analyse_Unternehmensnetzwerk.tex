\chapter{Analyse der Informationsquellen in einer typischen Unternehmensstruktur}
\label{cha:Analyse der Informationsquellen in einer typischen Unternehmensstruktur}

SIEM Systeme sind, vorallem in großen Unternehmensnetzwerken (siehe SANS Paper Proactive), zu einem Standard geworden um Informationen aus Logs verschiedener Systeme und anderen Kontextdaten zu gewinnen und diese in Szenarios einzuordnen. Um das Zwischenziel einer Bewertung der aktuellen Informationsmenge eines industriellen Netzwerkes zu erhalten bietet es sich durch diesen Zustand an, einen Abgleiches zwischen dem, im Bezug auf SIEM Technologie, etablierten Informationstand in Unternehmensnetzwerken und industriellen Netzwerken auszuführen. In diesem Kapitel soll es deshalb um eine akkurate Beschreibung der Informationsmenge eines typischen Unternehmensnetzwerkes gehen, die aus einer Analyse der gängigen Informationstypen und -kategorien erfolgt. Dazu wird im folgenden zunächst der Analyseansatz geschildert, der das Vorgehen schildert und die Analyse nachvollziehbar gestalten soll.

\section{Verwendete Analysemethode}

% Schritt 1: Recherchiere welche Log Dateien existieren
% Schritt 2: Jede Log Datei auf einem System ansehen und Eventformate heraussuchen
% Schritt 3: Events nach Typ kategorisieren
% Schritt 4: 
Die erste Frage für die Wahl des Analyseansatzes stellt sich sowohl in der gewählten Tiefe als auch in einer zielgerichteten Methodik. Das Ziel der Analyse ist es die sicherheitsrelevanten Infortmationsmenge zu erfassen. Der Begriff \glqq Sicherheitsrelevanz\grqq wird dazu für diese Thesis wie folgt definiert: Sicherheitsrelevante Informationen sind Informationen, die als \glqq Indicator of Compromise\grqq (IoC) dienen können. 
Ein IoC ist eine Information, die auf ein anomales Verhalten oder einen anomalen Zustand eines Netzwerkelementes schließen lässt. Um dieses Verhalten zu erkennen ergeben sich prinzipiell zwei Bereiche der Analyse: 1) Das Verhalten des Netzwerkelementes in sich, also der Zustand des Elements und Änderungen dieses Zustandes sowie 2) die Interaktion mit anderen Netzwerkelementen in Form des Austauschen von Nachrichten über die Netzwerkverbindung (vgl. Kaspersky Paper).
Diese Beobachtung wird durch das Vorgehen in Unternehmen durch den „Incident Response Process“ gestützt. Dieser Prozess dient als Durchführungsplan eines Teams, das für die Aufklärung und Behebung von Indikatoren und Brüchen des Sicherheitszustandes eines Netzwerkes  
zuständig ist. Der initiale Zustand dieses Plans ist die Überwachung der Netzwerkelemente und ihrer Kommunikation mit Hilfe von Detektions- und Analysesoftware (z.B. SIEM Systeme) und -hardware (z.B. \glqq Intrusion Detection Systems\grqq). Durch die Überwachung werden eine Vielzahl an Alarmen oder Sicherheitsereignissen produziert, die auf Fehler oder Anomalien im Netzwerk hinweisen. Werden die Informationen aus diesen Ereignissen als potentiell kritisch eingestuft, wird ein IoC an das Team ausgegeben, dass basierend auf Risiko und potentieller Schadensgröße weitere Untersuchungen einleitet. Bei diesen Untersuchungen wird der oder die IoC(s) als Ansatz genutzt. Ein IoC kann dabei auf Informationen im Payload oder Headern von Netzwerkpaketen als auch aus Logging Informationen eines Betriebssystems oder einer Überwachungssoftware auf einem Netzwerkelement beruhen. Basierend auf dem Ansatzpunkt können Informationen sowohl aus der Kommunikation als auch der Netzwerkelemente weitere Hinweise liefern um den potentiellen Sicherheitseinbruch einzugrenzen und schließlich zu identifizieren.
Aus diesem Grund ist es wichtig, beide Domänen zu untersuchen. Der Analyseansatz lehnt sich dabei an den Incident Response Prozess an. Zunächst wird der Zustand des Netzwerkes beschrieben und alle Netzwerkelemente sowie die potentiellen Kommunikationsschnittstellen erfasst. Darauf folgt die Untersuchung der Netzwerkelemente und ihrer Kommunikation in einer iterativen Vorgehensweise. Diese Vorgehensweise ermöglicht die schrittweise Vertiefung der Analyse  und soll als Ergebnis eine Übersicht pro Iterationsebene für beide Domänen erzielen. 
Die Analyse der Netzwerkelemente beginnt daher zunächst mit der Einteilung der Elemente in verschiedene Typen. Dabei wird unterschieden zwischen Servern mit verschiedenen Betriebssystemtypen sowie Netzwerkgeräte (Switches, Router) und Sicherheitselemente (Firewalls, Intrusion Detection Systeme). Um die Informationsmenge zu kategorisieren wird jeder Elementtyp auf die Existenz von Logdateien untersucht und die Struktur der Ereignisse in den Logdateien beschrieben. Eine Sammlung der typischen Quellen wird hinzugefügt um einen Überblick über den Ursprung typischer Ereignisse zu erreichen.
Für die Betrachtung der Kommunikation werden die verschiedenen Schnittstellen basierend auf den verwendeten Netzwerkprotokollen untersucht. Für die Klassifizierung wird basierend auf der Anwendung des OSI-Modells durchgeführt, welches als repräsentatives Kategorisierungsmodell verwendet wird. Diese Klassifizierung zeigt den allgemeinen Rahmen der erwarteten Informationsmenge pro Layer auf (vgl. TCP/IP Stack) und stellt eine Basis für die Analyse der verwendeten Felder der Protokollheader dar. 

Daraus folgt die weitere Struktur des Kapitels: Zunächst wird das definierte Model beschrieben, woraufhin zunächst die Netzwerkelemente und daraufhin die Netzwerkprotokolle betrachtet werden.

\todoQuestion{Sollen Angriffsvektoren /-szenarien hier mit eingepflegt werden oder im Vergleich integriert werden?}

\section{Beschreibung der Unternehmensarchitektur (Model)}

%Kommentar: Vielleicht macht es Sinn erst die Elemente und dann die Architektur zu beschreiben...

%Ziel 1: Beschreibung der ausgewählten Kernfaktoren basierend auf der Grundlagenrecherche
%Ziel 2: Beschreibung der Architektur & Interaktionen als Grundlage der Analyse

%Architekturidee:
%Router <--> Firewall <--> DMZ (Webserver mit Webservice) <--> Firewall <--> Intranet (User PC) <--> Firewall <--> Restricted Zone (Database Server & SIEM)

\todoForm{Das Ziel noch mehr herausstellen,Grundlegende Gedanke hinter der Erstellung des Models}
\todoForm{Einbinden, dass das Model auf der Basis von Netzwerkarchitekturen erstellt wurde} 
Der Aufbau des Netzwerkes orientiert sich an typischen Elementen, die in einem Unternehmensnetzwerk zu finden sind. Dabei besteht die Notwendigkeit das Model in seiner Größe und Komplexität einzugrenzen um eine Analyse in sinnvollem Maße zu ermöglichen. 
Die Architektur und Einteilung in verschiedene Netzwerkzonen ist dabei von üblichen Sicherheitszonen abgeleitet. Während die Positionierung der Elemente keine besondere Rolle spielt im Bezug auf die ermittelbaren Informationen pro Element ist jedoch anzumerken, dass auch diese Informationen, z.B. in Form der Auswertung von Netzwerkadressen, einem SIEM-System nützliche Informationen zur Verfügung stellen können. Die Architektur des Models beginnt an seinem \glqq Rand\grqq, dem Zugang zum WAN (Internet) über einen Gateway-Router. Dieser Router ermöglicht die Weiterleitung von Netzwerkpaketen in und aus der angrenzenden Netzwerkzone, der de-militarisierten Zone (DMZ). 
In der DMZ sind für diese Model ein Apache Webserver auf der Basis des Betriebssystems \glqq CentOS\grqq und ein E-Mail Server, bestehend aus dem Windows Webserver \glqq Internet Information Service\grqq (IIS) und dem darüber liegenden E-Mail Server \glqq Microsoft Exchange\grqq. Diese Netzwerkelemente bieten die Möglichkeit Datenverkehr über die Protokolle HTTP und die Microsoft Schnittstelle MAPI (bestehend aus RPC \& HTTP Datenverkehr) zu untersuchen, sowie Informationen aus Log-Dateien sowohl von einem Linux-basierten Server als auch von einem Windows-basierten Server zu analysieren. Alle Elemente werden durch \glqq Managed Switches\grqq miteinander verbunden.
\todoQuestion{Remote Access Server in Form von VPN, Protokolle? Traffic? Sinnvoll?}

Die DMZ wird von der nächsten Zone, dem Extranet, durch eine Firewall überwacht, die eingehenden initiale Kommunikation blockiert. Firewalls gehören zu den Grundlagen der Sicherheit von Unternehmen und werden häufig platziert um Datenverkehr in und aus bestimmten Netzwerkzonen zu überwachen. Innerhalb der Netzwerkzone werden zwei Benutzer-PCs eingesetzt. Diese werden platziert um Abweichungen und andere Benutzerinteraktionen in die erhaltbare Informationsmenge zu integrieren. Die beiden Computer unterscheiden sich wie auch die Server der DMZ im Betriebssystem, sodass ein PC auf Basis von Windows und damit verbundener Benutzerverwaltung durch Active Directory enthalten ist, sowie ein PC mit einem Linux-basierten Betriebssystem. Zudem wird ein Netzwerkdrucker integriert.
\todoQuestion{Gehören Printserver hier rein? Welche Zusatzinformationen bringt ein Printserver?}

Als dritte Netzwerkzone wird als \glqq Restricted Area\grqq bezeichnet. Innerhalb dieser Netzwerkzone werden verschiedene Windows-basierte Server eingesetzt um die Funktionalitäten eines Unternehmensnetzwerk zu simulieren. Die verschiedenen Servertypen wurden ausgewählt um erhaltbare Informationen aus unterschiedlichen, typisch genutzten Netzwerkprotokollen aufzuzeigen. Dazu gehören: 
\begin{itemize}
\item ein Microsoft SQL Datenbankserver 
\item ein Active Directory Domain Controller 
\item ein FTP-Server
\end{itemize}

Zusätzlich ist in dieser Zone als Ergänzung auch der SIEM-Server gesetzt.

Zwischen den verschiedenen Elementen des Netzwerkes werden Informationen ausgetauscht z.B. für die Abfrage einer Ressource, Herstellung einer Kommunikation oder Übermittlung von Authentifizierungsinformationen. Dieser Austausch wird durch verschiedene Protokolle gesteuert. Der Inhalt der gesendeten Datenpakete (Payload) wird mit verschiedenen Header-Informationen gekapselt. Neben den grundlegenden Header-Daten der Protokolle auf niedrigeren Ebenen (Ethernet, IP, TCP/UDP) sollen in diesem auch verschiedene Protokolle der Ebene 7 betrachtet werden. Die zugehörigen Header-Information sind spezifisch für das entsprechende Protokoll und können z.B. Informationen über den Status der Kommunikation beinhalten. Diese Informationen sind bei der Analyse von Netzwerkdaten nützlich um den Kontext der ausgetauschten Daten zu verstehen und einen Ablauf der Kommunikation nachzuvollziehen.
In dem verwendeten Model werden verschiedene Protokolle bei der Kommunikation zwischen den verschiedenen Komponenten betrachtet. Die Wahl des Protokolls hängt von der spezifischen Schnittstelle ab. Eine Auflistung der Schnittstellen für das Ansprechen der entsprechenden Komponente werden in der folgenden Tabelle aufgelistet:

\todoForm{Tabelle der Kommunikationsschnittstellen einfügen, Name, Protokoll, Kurzbeschreibung}
Kommunikationsschnittstellen:
\begin{itemize}
\item Windowsserver: RDP
\item Linuxserver: SSH
\item Microsoft SQL Datenbank: SQL
\item Active Directory: LDAP
\item FTP-Server: FTP
\item Firewall: SSH / HTTP
\item WebServer: HTTP
\item Exchange: MAPI (RPC + HTTP)
\end{itemize}

Die Betrachtung der Informationen von den Netzwerkelementen und der Kommunikation zwischen den Elementen ergibt das Gesamtbild der ermittelbaren Informationsmenge in diesem Model. Die folgenden Schritte betrachten beide Teile für die jeweils zu untersuchenden Elemente.

\subsection{Systeme}

%Frage: Warum lasse ich Elemente wie z.B. eine zentrale Benutzerverwaltung raus? Ist das nicht eigentlich relevant?
%Antwort: Die zentrale Frage hier ist: Welche Daten bekomme ich, im Bezug auf den normalen PC, aus einer Benutzerverwaltung wie z.B. einem Active Directory, die ich nicht aus einem Windowssystem bekomme?
%Um diese Frage zu beantworten müsste ich wissen nach welchen Informationen ich suche im Bezug auf die Benutzerverwaltung?
%Brainstorming: Login/Logoff von Benutzern (beide), Berechtigungen des Benutzers (beide), Zertifizierung? (nur AD), 
%Zentrale Frage des IAMs: Wer (Login) hat was (Lokale Prozessevents, Netzwerkdaten) wann (Timestamps, generell) wo (Kombination der Elemente) wie (Kombination der Elemente) gemacht?
%Spezifizierte Frage: Welche Kontextinformationen bekomme ich von einer zentralen Benutzerverwaltung, die ich nicht von einem lokalen System bekomme?
%Ansatz: Erstmal Windows anschauen, dann ggf. nach AD suchen
%Temporäres Resultat: AD erstmal weglassen, später einbauen


%Frage: Brauche ich Remote-Access?

\begin{itemize}
\item Router, Rolle: Zugang zum WAN
\item Switch, Rolle: Verbindungen zwischen den Elementen herstellen (Router <-> Switch <-> Firewall...)
\item Firewall, Rolle: Untersuchung des Netzwerkdatenverkehrs 
\item Webserver (Apache), Rolle: Plattform für den Webservice (Shop)
\item Endbenutzer PC, Rolle: Element über das Benutzerzugriffe auf verschiedene Systeme ausgeführt werden
\item Datenbankserver (MSSQL), Rolle: Enthält Businesslogik und kritische Daten
\item IDS (NIDS)?, Rolle (potentiell): Sicherheitssystem mit Event-Daten für das SIEM
\item SIEM, Rolle: Zentrale Log-Verwaltung und Korrelation 
\end{itemize}
\subsection{Applikationen}

%Die Frage ist, welche Applikationen brauche ich?
%#1 Eine Applikation, die auf das Dateisystem eines Elementes zugreift (WebServer)
%#2 Eine Applikation, die über das Netzwerk kommuniziert
%#3 Eine Applikation die Log-Daten erzeugt

%//Nicht zu komplex machen

\subsection{Interaktionen der Systeme (und Anwender)}

% Grobe Übersicht, erster Ansatz
% Anwender <--> Webserver
% WebService <--> Webserver
% Router(GW) <--> Webserver
% Webserver <--> Datenbank Server
% Endbenutzer <--> Datenbank Server
% Endbenutzer <--> Webserver
% Endbenutzer <--> Router(GW)

% Wo überwacht das NIDS (inline?)?
% Was und wie extrahiert das SIEM?



\section{Resultierende Informationstypen \/ -kategorien}
\subsection{Extraktion und Verarbeitung}
\subsection{Windows}
\subsection{Linux}
\subsection{Beispielapplikation}
\subsection{Netzwerk}
\section{Analyse im Bezug auf Angriffsvektoren}