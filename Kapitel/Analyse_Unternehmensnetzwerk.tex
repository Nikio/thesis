\chapter{Analyse der Informationsquellen in einer typischen Unternehmensstruktur}
\label{cha:Analyse der Informationsquellen in einer typischen Unternehmensstruktur}

% Netzwerk so gut es geht nachbauen =)

\section{Beschreibung der Unternehmensarchitektur (Beispiel)}

%Kommentar: Vielleicht macht es Sinn erst die Elemente und dann die Architektur zu beschreiben...

%Ziel 1: Beschreibung der ausgewählten Kernfaktoren basierend auf der Grundlagenrecherche
%Ziel 2: Beschreibung der Architektur & Interaktionen als Grundlage der Analyse

%Architekturidee:
%Router <--> Firewall <--> DMZ (Webserver mit Webservice) <--> Firewall <--> Intranet (User PC) <--> Firewall <--> Restricted Zone (Database Server & SIEM)

/Das Ziel noch mehr herausstellen, was ist der Grundlegende Gedanke hinter der Erstellung des Models? 
//Einbinden, dass das Model auf der Basis von Netzwerkarchitekturen erstellt wurde
„Der Aufbau des Netzwerkes orientiert sich an typischen Elementen, die in einem Unternehmensnetzwerk zu finden sind. Dabei besteht die Notwendigkeit das Model in seiner Größe und Komplexität einzugrenzen um eine Analyse in sinnvollem Maße zu ermöglichen. Die Architektur und Einteilung in verschiedene Netzwerkzonen ist dabei von üblichen Sicherheitszonen abgeleitet. Während die Positionierung der Elemente keine besondere Rolle spielt im Bezug auf die ermittelbaren Informationen pro Element ist jedoch anzumerken, dass auch diese Informationen, z.B. in Form der Auswertung von Netzwerkadressen, einem SIEM-System nützliche Informationen zur Verfügung stellen können. Die Architektur des Models beginnt an seinem „Rand“, dem Zugang zum WAN (Internet) über einen Gateway-Router. Dieser Router ermöglicht die Weiterleitung von Netzwerkpaketen in und aus der angrenzenden Netzwerkzone, der de-militarisierten Zone (DMZ). In der DMZ sind für diese Model ein Apache Webserver auf der Basis des Betriebssystems „CentOS“ und ein E-Mail Server, bestehend aus dem Windows Webserver „Internet Information Service“ (IIS) und dem darüber liegenden E-Mail Server „Microsoft Exchange“. Diese Netzwerkelemente bieten die Möglichkeit Datenverkehr über die Protokolle HTTP und die Microsoft Schnittstelle MAPI (bestehend aus RPC & HTTP Datenverkehr) zu untersuchen, sowie Informationen aus Log-Dateien sowohl von einem Linux-basierten Server als auch von einem Windows-basierten Server zu analysieren. Alle Elemente werden durch „Managed Switches“ miteinander verbunden.
//Remote Access Server in Form von VPN, Protokolle? Traffic? Sinnvoll?

Die DMZ wird von der nächsten Zone, dem Extranet, durch eine Firewall überwacht, die eingehenden initiale Kommunikation blockiert. Firewalls gehören zu den Grundlagen der Sicherheit von Unternehmen und werden häufig platziert um Datenverkehr in und aus bestimmten Netzwerkzonen zu überwachen. Innerhalb der Netzwerkzone werden zwei Benutzer-PCs eingesetzt. Diese werden platziert um Abweichungen und andere Benutzerinteraktionen in die erhaltbare Informationsmenge zu integrieren. Die beiden Computer unterscheiden sich wie auch die Server der DMZ im Betriebssystem, sodass ein PC auf Basis von Windows und damit verbundener Benutzerverwaltung durch Active Directory enthalten ist, sowie ein PC mit einem Linux-basierten Betriebssystem. Zudem wird ein Netzwerkdrucker integriert.
//Gehören Printserver hier rein? Welche Zusatzinformationen bringt ein Printserver?

Als dritte Netzwerkzone wird als „Restricted Area“ bezeichnet. Innerhalb dieser Netzwerkzone werden verschiedene Windows-basierte Server eingesetzt um die Funktionalitäten eines Unternehmensnetzwerk zu simulieren. Die verschiedenen Servertypen wurden ausgewählt um erhaltbare Informationen aus unterschiedlichen, typisch genutzten Netzwerkprotokollen aufzuzeigen. Dazu gehören: 
\begin[itemize]
\item ein Microsoft SQL Datenbankserver 
\item ein Active Directory Domain Controller 
\item ein FTP-Server
\end[itemize]

Zusätzlich ist in dieser Zone als Ergänzung auch der SIEM-Server gesetzt.“

\subsection{Systeme}

%Frage: Warum lasse ich Elemente wie z.B. eine zentrale Benutzerverwaltung raus? Ist das nicht eigentlich relevant?
%Antwort: Die zentrale Frage hier ist: Welche Daten bekomme ich, im Bezug auf den normalen PC, aus einer Benutzerverwaltung wie z.B. einem Active Directory, die ich nicht aus einem Windowssystem bekomme?
%Um diese Frage zu beantworten müsste ich wissen nach welchen Informationen ich suche im Bezug auf die Benutzerverwaltung?
%Brainstorming: Login/Logoff von Benutzern (beide), Berechtigungen des Benutzers (beide), Zertifizierung? (nur AD), 
%Zentrale Frage des IAMs: Wer (Login) hat was (Lokale Prozessevents, Netzwerkdaten) wann (Timestamps, generell) wo (Kombination der Elemente) wie (Kombination der Elemente) gemacht?
%Spezifizierte Frage: Welche Kontextinformationen bekomme ich von einer zentralen Benutzerverwaltung, die ich nicht von einem lokalen System bekomme?
%Ansatz: Erstmal Windows anschauen, dann ggf. nach AD suchen
%Temporäres Resultat: AD erstmal weglassen, später einbauen


%Frage: Brauche ich Remote-Access?

\begin{itemize}
\item Router, Rolle: Zugang zum WAN
\item Switch, Rolle: Verbindungen zwischen den Elementen herstellen (Router <-> Switch <-> Firewall...)
\item Firewall, Rolle: Untersuchung des Netzwerkdatenverkehrs 
\item Webserver (Apache), Rolle: Plattform für den Webservice (Shop)
\item Endbenutzer PC, Rolle: Element über das Benutzerzugriffe auf verschiedene Systeme ausgeführt werden
\item Datenbankserver (MSSQL), Rolle: Enthält Businesslogik und kritische Daten
\item IDS (NIDS)?, Rolle (potentiell): Sicherheitssystem mit Event-Daten für das SIEM
\item SIEM, Rolle: Zentrale Log-Verwaltung und Korrelation 
\end{itemize}
\subsection{Applikationen}

%Die Frage ist, welche Applikationen brauche ich?
%#1 Eine Applikation, die auf das Dateisystem eines Elementes zugreift (WebServer)
%#2 Eine Applikation, die über das Netzwerk kommuniziert
%#3 Eine Applikation die Log-Daten erzeugt

%//Nicht zu komplex machen

\subsection{Interaktionen der Systeme (und Anwender)}

% Grobe Übersicht, erster Ansatz
% Anwender <--> Webserver
% WebService <--> Webserver
% Router(GW) <--> Webserver
% Webserver <--> Datenbank Server
% Endbenutzer <--> Datenbank Server
% Endbenutzer <--> Webserver
% Endbenutzer <--> Router(GW)

% Wo überwacht das NIDS (inline?)?
% Was und wie extrahiert das SIEM?

\section{Verwendete Analysemethode}

% Schritt 1: Recherchiere welche Log Dateien existieren
% Schritt 2: Jede Log Datei auf einem System ansehen und Eventformate heraussuchen
% Schritt 3: Events nach Typ kategorisieren
% Schritt 4: 

\section{Resultierende Informationstypen \/ -kategorien}
\subsection{Extraktion und Verarbeitung}
\subsection{Windows}
\subsection{Linux}
\subsection{Beispielapplikation}
\subsection{Netzwerk}
\section{Analyse im Bezug auf Angriffsvektoren}