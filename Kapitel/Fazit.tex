\chapter{Fazit}
\label{cha:Fazit}

Die Analyse der Netzwerke und die Durchführung des Vergleiches haben Unterschiede in den verfügbaren Informationsbereichen aufgezeigt. Die Informationsmengen wurden nach Systemen und Kommunikationsprotokollen unterteilt und basierend auf ihrer Zugehörigkeit zu definierten Kategorien verglichen. Der Vergleich offenbarte Unterschiede in der Detailtiefe der verfügbaren Informationen, u.a. im Bereich der Dokumentation von Parameter- und Bausteinänderungen der untersuchten SPS. Dies konnte durch eine Analyse des Stuxnet Angriffes als signifikante Informationen für die Erkennung und Eingrenzung eines Malware-Angriffes mit Charakteristiken des Stuxnet-Angriffes bzgl. der Manipulation von SPSen eingestuft werden. Ein Lösungsansatz, der eine Prüfung von Änderungen bestimmter Daten auf der Grundlage von Verdachtsereignissen darstellt, wurde beschrieben und demonstriert. 

Zusammenfassend konnte die ursprüngliche These für die gewählte Modellbasis belegt werden. Für eine allgemeinere Aussage müssten allerdings weitere Analysen bzgl. anderer Systemkombination und Hersteller vorgenommen werden. Erfolgversprechend könnte dies etwa für eine Analyse weiterer SPSen der Firma Siemens oder konkurrierender Produkte (z.B. von Rockwell Automation) bzgl. der gefundenen Unterschiede sein. 

Die Ausführung des Vergleiches hat Schwierigkeiten im Bereich der Herstellung von Beziehungen zwischen den verfügbaren Daten und der Zuordnung zu den Kategorien gezeigt. Mögliche Folgearbeiten könnten es sich zur Aufgabe machen, einen genaueren Blick auf einzelne Kategorien zu werfen, weitere Unterschiede in den Details zu finden und diese auf Signifikanz bzgl. dokumentierter Angriffe auf industrielle Fertigungssysteme zu prüfen. 
Eine weitere interessante Folgefrage stellt sich auch im Bezug der Signifikanz der gefundenen Unterschiede in Bezug auf weitere bekannte Angriffsszenarien und potentielle Erkenntnisse und weitere Lösungsansätze für die Schließung der Informationslücken.


